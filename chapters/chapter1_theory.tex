The Standard Model of particle physics (\gls{SM}) \cite{SM} is a relativistic quantum field theory (QFT) describing the interactions of all known fundamental constituents of matter. It comprises the strong and electroweak interactions, mediated by gauge bosons.  In the formalism of \gls{QFT} classical fields are quantized and the information of the interaction of the particles are encoded in the lorentz-invariant Lagrangian density $\mathcal{L}$. 


\section{Particles and interactions}
\label{sec:section_1_1}

In the \gls{SM} elementary particles can be characterized by their spin \cite{SPIN}, which is a quantum number describing the intrinsic angular momentum. Fermions carry spin $\frac{1}{2}$ and are represented by Dirac spinors $\psi$, which are solving the Dirac equation. The equation of motion are derived from the Lagrangian density. The requirement of the invariance of the Lagrangian under local gauge transformation leads to the introduction of gauge bosons, interacting with the fermions. Gauge bosons are elementary particles with spin 1. All relevant symmetries and the corresponding interactions are described in the following, the schematic overview of all elementary particles and their properties are shown in figure \ref{fig:fig_1_1}.

\subsection{U(1)${_{\text{QED}}}$ symmetry and quantum electrodynamics (\gls{QED})}
\label{sec:section_1_1_1}

The Lagrangian of a fermion \cite{Peskin}, with $\gamma^{\mu}$ as the gamma matrices and $\bar{\psi} = \psi^{\dagger}\gamma^{0}$, can be written as:

\begin{equation}
	\label{eq:eq_1_1}
	\mathcal{L}_{0} =  i\bar{\psi}(x)\gamma^{\mu}\partial_{\mu}\psi(x) - m\bar{\psi}(x)\psi(x).
\end{equation}

Performing a global transformation of the kind $\psi(x) \rightarrow \psi'(x) = e^{iQ\alpha}\psi(x)$, with Q and $\alpha$ as real constants, the Lagrangian stays invariant under this so called U(1) transformation \cite{SM}. The invariance is not fulfilled if $\alpha$ depends explicitly on the space-time $x$, in this case a local gauge transformation is performed. To ensure invariance under a local gauge transformation, a 4-component Lorentz vector field $A_{\mu}(x)$ is introduced with the following property under U(1) transformation: $A_{\mu} \rightarrow A'_{\mu}(x) = A_{\mu}(x) + \frac{1}{e}\partial_{\mu}\alpha(x)$. Using the definition of the covariant derivative $D_{\mu} = \partial_{\mu} - ieQA_{\mu}(x)$, the following Lagrangian stays invariant under local U(1) transformation:

\begin{equation}
	\label{eq:eq_1_2}
	\mathcal{L} = i\bar{\psi}(x)\gamma^{\mu}D_{\mu}\psi(x) - m\bar{\psi}(x)\psi(x) = \mathcal{L}_{0} + eQA_{\mu}(x)\bar{\psi}(x)\gamma^{\mu}\psi(x).
\end{equation}

With the gauge transformation the gauge boson of the electromagnetic interaction is introduced, which couples to the fermions with electric charge e: the photon $\gamma$.


\subsection{SU(3)${_C}$ symmetry and quantum chromodynamics (\gls{QCD})}
\label{sec:section_1_1_2}

Quarks are fermions carrying one of three possible colour charges denoted with red (r), blue (b) and green (g). Labelling the quark spinor with colour $c$ as $q^{c} = \psi_{q}(c)$ and putting them all into a three vector $q_{f} = (q^{b}, q^{r}, q^{g})$, the Lagrangian of the quarks can be written as

\begin{equation}
	\label{eq:eq_1_3}
	\mathcal{L}_{0} = \sum_{f} \bar{q_{f}}(i\gamma^{\mu}\partial_{\mu} - m_{f})q_{f}.
\end{equation}

The index $f$ denotes the flavour of the quark is explained below. Like in the \gls{QED} example a local gauge theory exists, which introduces gauge bosons. A transformation of the kind $q_{f} \rightarrow q'_{f} = e^{i\frac{\lambda^{a}}{2}\alpha_{a}(x)} q_{f}$, with $\frac{1}{2} \lambda^{a}$ as non commuting 3x3 matrices and $a \in (1, 2, .., 8)$, is a local SU(3)$_{C}$ transformation \cite{QCD}. To get a Lagrangian invariant under this transformation, 8 new gauge bosons $G^{\mu}_{a}(x)$, the gluons, have to be introduced, corresponding to the number of generators $\lambda^{a}$ to represent the SU(3)$_{C}$ algebra. Defining the covariant derivative $D_{\mu} = \partial_{\mu}-ig_{s}\frac{\lambda^{a}}{2}G^{\mu}_{a}(x)$ and demand the gluons to transform using $U = e^{i\frac{\lambda^{a}}{2}\alpha_{a}(x)}$ like $G^{\mu}_{a}(x) \rightarrow G^{\mu}(x)' = UG^{\mu}U^{\dagger} - \frac{i}{g_s}(\partial^{\mu}U)U^{\dagger}$ the invariant Lagrangian of \gls{QCD} can be written like 

\begin{equation}
	\label{eq:eq_1_4}
	\mathcal{L} = \sum_{f} \bar{q_{f}}(i\gamma^{\mu}D_{\mu} - m_{f})q_{f}.
\end{equation}

\subsection{SU(2)$_{W}$ x U(1)$_{Y}$ symmetry, the electroweak interaction and Higgs mechanism}
\label{sec:section_1_1_3}

The flavour of fermions is a quantum number describing the particle type. Two types of fermions of exists, quarks and colourless fermions, called leptons. The quark flavours are called up, down, charm, strange, top, bottom and the lepton flavours are called electron, muon and tau. Beside the flavour the chirality is a key quantity in the electroweak theory, because not all chirality configuration of the fermions couples to the gauge bosons. Defining the spinors of fermions with left-handed(L) or right-handed(R) chirality with $\psi_{L/R} = \frac{1}{2}(1\mp \gamma^{5})\psi$ and for simplicity looking only at the first generation of quarks and leptons, the spinor of quarks and leptons are defined as $u = \psi_{u}, d = \psi_{d}, e = \psi_{e}, \nu = \psi_{\nu_{e}}$. If the fermions are assembled in left-handed doublets and right handed singlets, as $q_{L} = (u_{L}, d_{L}), u_{R}, d_{R}, l_{L} = (e_{L}, \nu_{L}), e_{R}$ and considered to be massless, the Lagrangian of the electroweak theory is

\begin{equation}
	\label{eq:eq_1_5}
	\mathcal{L}_{0} = i\bar{q}_{L}\gamma^{\mu}\partial_{\mu}q_{L} + i\bar{l}_{L}\gamma^{\mu}\partial_{\mu}l_{L} + \sum_{f \in (e, u, d)} i\bar{f}_{R}\gamma^{\mu}\partial_{\mu}f_{R}.
\end{equation}

The local SU(2)$_{W}$ x U(1)$_{Y}$ gauge transformation \cite{EWK}, which is a transformation of the kind

\begin{equation}
	\label{eq:eq_1_6}
	\begin{split}
		q_{L} \rightarrow q'_{L} = e^{iQ_{q} \alpha_{Y}(x)} e^{i\frac{\sigma_{j}}{2}\alpha^{j}_{W}(x)}q_{L} \\
		l_{L} \rightarrow l'_{L} = e^{iQ_{l} \alpha_{Y}(x)} e^{i\frac{\sigma_{j}}{2}\alpha^{j}_{W}(x)}l_{L} \\
		f_{R} \rightarrow f'_{R} = e^{iQ_{f} \alpha_{Y}(x)}, \quad f \in (e, u, d)
	\end{split}			
\end{equation}

with $\sigma_{j}, j \in (1,2,3)$ as the Pauli matrices, $Q$ as real constant and $\alpha_{Y}, \alpha^{j}_{W}$ as real function of space-time, introduces 4 gauge bosons, the three $W^{j}_{\mu}(x)$, originating from SU(2)$_{W}$, and the $B_{\mu}$ from U(1)$_{Y}$. Using $U_{L} = e^{i\frac{\sigma_{j}}{2}\alpha^{j}_{W}(x)}$, demanding the transformation properties $B_{\mu} \rightarrow B'_{\mu} = B_{\mu} + \frac{1}{g'}\partial_{\mu}\alpha_{Y}$, $W'_{\mu} \rightarrow U_{L}\frac{\sigma_{j}}{2}W_{\mu}^{j}U_{L}^{\dagger} - \frac{i}{g}(\partial_{\mu}U_{L})U_{L}^{\dagger}$ and defining the covariant derivative 

\begin{equation}
	\label{eq:eq_1_7}
	\begin{split}
		D^{q}_{\mu} = \partial_{\mu}  - ig\frac{\sigma_{j}}{2}W^{j}_{\mu} - ig'Q_{q}B_{\mu} \\
		D^{l}_{\mu} = \partial_{\mu}  - ig\frac{\sigma_{j}}{2}W^{j}_{\mu} - ig'Q_{l}B_{\mu} \\
		D^{f_R}_{\mu} = \partial_{\mu}  - ig'Q_{f}B_{\mu}, \quad f \in (e, u, d)
	\end{split}
\end{equation}

the gauge invariant Lagrangian of the electroweak theory can be written as 

\begin{equation}
	\label{eq:eq_1_8}
	\mathcal{L} = i\bar{q}_{L}\gamma^{\mu}D_{\mu}^{q}q_{L} + i\bar{l}_{L}\gamma^{\mu}D_{\mu}^{l}l_{L} + \sum_{f \in (e, u, d)} i\bar{f}_{R}\gamma^{\mu}D^{f_R}_{\mu}f_{R}.
\end{equation}

The problem from this formulation arises from the fact that the gauge bosons and fermions are measured to have a mass. \\

To keep the local gauge invariance and allow particles to acquire mass, the Higgs mechanism \cite{HIGGS} is introduced. Considering a complex scalar field doublet $\Phi(x) = (\phi^{1}(x), \phi^{2}(x))$, which transform under SU(2)$_{W}$ like $\Phi(x) \rightarrow \Phi'(x) = e^{iQ_{\Phi} \alpha_{Y}(x)} e^{i\frac{\sigma_{j}}{2}\alpha^{j}_{W}(x)}\Phi(x)$ the Lagrangian can be written as

\begin{equation}
	\label{eq:eq_1_9}
	\mathcal{L}_{H} = (D^{\Phi}_{\mu}\Phi)(D^{\Phi, \mu}\Phi)^{\dagger} - \mu^2\Phi^{\dagger}\Phi + \frac{1}{2}(\Phi^{\dagger}\Phi)^2, \quad \mu^2 > 0
\end{equation}

with the covariant derivative $D^{\Phi}_{\mu} = \partial_{\mu}  - ig\frac{\sigma_{j}}{2}W^{j}_{\mu} - ig'Q_{\Phi}B_{\mu}$. Because of the condition for $\mu$, the vacuum expectation value of $\Phi$ is non vanishing: $\Phi\Phi^{\dagger} = \frac{\mu^2}{\lambda} \equiv \nu$. One of the four degrees of freedom of $\Phi$ is set with this condition, the other three degrees of freedom can be removed by a gauge transformation. Choosing the unitary gauge, $\Phi$ can be written as $\Phi = (0, \nu + \frac{1}{\sqrt{2}}H(x))$, with $H(x)$ as the scalar spin 0 field, called Higgs boson. Plugging this into equation \ref{eq:eq_1_9}, the following relation show up

\begin{equation}
	\label{eq:eq_1_10}
	\begin{split}
		Z_{\mu} = \cos(\theta_{W})W^{3}_{\mu} + \sin(\theta_{W})B_{\mu}, \quad M_{Z}^2 = \frac{1}{2}(g^2 + g'^2)\nu^2 \\
		A_{\mu} = -\sin(\theta_{W})W^{3}_{\mu} + \sin(\theta_{W})B_{\mu}, \quad M_{A} = 0 \\
		W^{\pm}_{\mu} = \frac{1}{\sqrt{2}}W^{1}_{\mu} \mp W^{2}_{\mu}, \quad M_{W}^2 = \frac{1}2{\sqrt{2}}(g^2+g'^2)\nu^2
	\end{split}
\end{equation}

The $Z$ boson and the $W^{\pm}$ from the weak interaction appeared with the introduction of the Higgs mechanism \cite{HIGGDISCOVER}, which are both massive, like measured in nature. Also the photon $A_{\mu}$, which also came up during the \gls{QED} discussion, appeared with zero mass. 


\begin{figure}[hpt]
	\centering
	\includegraphics[width=0.8\textwidth]{pictures/Standard_Model_of_Elementary_Particles.pdf}

	\caption[Overview of Standard Model particles]{Overview of all known elementary particles of the \gls{SM} with their measured properties, taken from \cite{SMPARTICLES}}
	\label{fig:fig_1_1}
\end{figure}


\section{Lepton flavour violation}
\label{sec:section_1_3}

In the \gls{SM} two types of quantum numbers are assigned to leptonic fermions. The lepton number $L$ assigns the value 1/-1 to all leptonic fermions/anti fermions, and zero to everything else. In comparison to that the lepton flavour $L_{\ell}$ with $\ell \in (e, \mu, \tau)$ assigns 1/-1 to the lepton/anti lepton with flavour $\ell$, zero to everything else. A overview is shown in table \ref{tab:tab_1_2}


\begin{table}[h]
	\centering
	\caption[Lepton number/lepton flavour of leptons]{Lepton number/lepton flavour for all known leptons in the \gls{SM}}
	\label{tab:tab_1_2}

	\begin{tabular}{l|l|l|l|l}
		Lepton				&$L$		&$L_{e}$	&$L_{\mu}$	&$L_{\tau}$	\\ \hline
		
		$e^{-}/e^{+}$			&$\pm 1$	&$\pm 1$	&0		&0		\\

		$v_{e}/\bar{v}_{e}$		&$\pm 1$	&$\pm 1$	&0		&0		\\
		
		$\mu^{-}/\mu^{+}$		&$\pm 1$	&0		&$\pm 1$	&0		\\

		$v_{\mu}/\bar{v}_{\mu}$		&$\pm 1$	&0		&$\pm 1$	&0		\\
		
		$\tau^{-}/\tau^{+}$		&$\pm 1$	&0		&0		&$\pm 1$	\\

		$v_{\tau}/\bar{v}_{\tau}$	&$\pm 1$	&0		&0		&$\pm 1$	\\			
	\end{tabular}
\end{table}


\subsection{Global lepton flavour symmetry}
\label{sec:section_1_3_1}

The \gls{SM} Langragian $\mathcal{L}^{\ell}_{\text{SM}}$ \cite{Peskin, EWK}, only looking at the parts where leptons are involved, and after introducing the Higgs bosons like discussed in section \ref{sec:section_1_1}, can be written as

\begin{multline}
	\label{eq:eq_1_12}
	\mathcal{L}^{\ell}_{\text{SM}} = \sum_{f \in (e, \mu, \tau)} \underbrace{-(1+\frac{H}{\nu}) m_{f} \bar{f}f}_\text{Higgs coupling} + \underbrace{\frac{g}{\sqrt{2}} (W^{+}_{\mu} \bar{\nu}^{f}_{L} \gamma^{\mu} f_{L} + W^{-}_{\mu} \bar{f}_{L} \gamma^{\mu} \nu^{f}_{L})}_{W^{\pm} \text{ coupling}} \\
	+ \underbrace{\frac{g}{\cos{\theta_{W}}}(Z_{\mu} \frac{1}{2} \bar{\nu}^{f}_{L} \gamma^{\mu}\nu^{f}_{L} + \bar{f}_{L}(-\frac{1}{2} + \sin^2{\theta_{W}})f_{L} + \bar{f}_{R}(\sin^2{\theta_{W}})f_{L}}_{\text{Z coupling}} - \underbrace{eA_{\mu}\bar{f}\gamma^{\mu}f}_{\text{photon coupling}}.
\end{multline}

This Langrangian is invariant under global flavour transformation U(1)$_{e}$ x U(1)$_{\mu}$ x U(1)$_{\tau}$ \cite{LFV1, LFV2}, which is a transformation of kind 

\begin{equation}
	\label{eq:eq_1_13}
	\begin{split}
		f \rightarrow f' = e^{i\alpha_{f}Q_{f}}f, \quad f \in (e, \mu, \tau) \\
		\nu^{f} \rightarrow \nu'^{f} = e^{i\alpha_{f}Q_{f}}\nu^{f}, \quad f \in (e, \mu, \tau) \\
	\end{split}
\end{equation}

with $Q_{f}$ and $\alpha_{f}$ as real constants. In comparison to the gauge symmetries this symmetry introduces no new gauge boson, a space-time dependent $\alpha_{f}$ would spoil the symmetry and cannot be fixed by a gauge boson introduction.

\subsection{Breaking of the symmetry due to neutrino masses}
\label{sec:section_1_3_2}

The Lagrangian in equation \ref{eq:eq_1_12} does not include mass terms for neutrinos, because of non-observed $\nu_R$, which leads to a vanishing neutrino mass term $-m_{\nu^{f}}\bar{\nu}^{f}\nu^{f} = -m_{\nu^{f}_{L}}(\bar{\nu}^{f}_{R}\nu^{f}_{L} + {\nu}^{f}_{L}\bar{\nu}^{f}_{R}) = 0$. The measurement of neutrino oscillation \cite{NEUTRINOOSC} requires neutrinos to be massive particles. This phenomena describe change of lepton flavour of the neutrinos, which is possible because flavour eigenstates are not equal to the mass eigenstates $|\nu_{f}> \neq |\nu_{m_{f}}>$ and are connected via the unitary 3x3 PMNS matrix $U$ \cite{PMNS}: $|\nu_{f}> =  \sum_{i} U^{i}_{f}|\nu_{m_{i}}>$. The probability P of the transition from flavour $\alpha$ to flavour $\beta$ \cite{NEUTRINOPROB} is given by 

\begin{equation}
	\label{eq:eq_1_14}
	P(\alpha \rightarrow \beta) = |\sum_{i, j} U_{\alpha}^{i} U_{\beta}^{j} e^{i\Delta m^2_{ij}L/(2E)}|^2.
\end{equation}

There are two possible ways to introduce neutrino mass terms in an extended \gls{SM} Lagrangian. One way is to postulate the existence right-handed neutrinos $\nu_R$, the second way is to demand that neutrinos are their own anti-particles. For both possibilities the most general Lagrangian including neutrino mass terms can be written as
\begin{equation}
	\label{eq:eq_1_15}
	\begin{split}
		\mathcal{L}_{\text{Dirac}} = -\bar{\nu}^{\ell'}_{R}M^{\text{Dirac}}_{\ell'\ell} \nu^{\ell}_{L} + \text{h.c.} \\
		\mathcal{L}_{\text{Majorana}} = -C(\bar{\nu}^{\ell'}_{L})^{T}M^{\text{Majorana}}_{\ell'\ell} \nu^{\ell}_{L} + \text{h.c.} 
	\end{split}
\end{equation}

with $M_{\ell'\ell}$ as complex 3x3 matrix, which is not diagonal, because of the different mass/flavour eigenstates of neutrinos. This leads to the breaking of the flavour symmetry, because $\mathcal{L}_{\text{Dirac}}$/$\mathcal{L}_{\text{Majorana}}$ are not invariant under the flavour transformation in equation \ref{eq:eq_1_13}. Neutrino masses introduces lepton flavour violation (\gls{LFV}), and branching ratios on neutrino mixing induced \gls{LFV} Z boson decays \cite{NEUTRINOLFV} can be calculated to be 

\begin{equation}
	\label{eq:eq_1_16}
	\begin{split}
		\text{BR}^{\nu \text{ mix}}(Z\to e\mu) < 10^{-55} \\
		\text{BR}^{\nu \text{ mix}}(Z\to e\tau) < 10^{-54} \\
		\text{BR}^{\nu \text{ mix}}(Z\to \mu\tau) < 10^{-60}
	\end{split}
\end{equation}

These branching ratios are far away from being detectable, which leads to a good approximation of conservation of lepton flavour, even with massive neutrinos.

\subsection{Lepton flavour violation beyond the Standard Model}
\label{sec:section_1_3_3}

The discussion in section \ref{sec:section_1_3_2} showed that lepton flavour symmetry is not absolute and can be broken. In the case of neutrino mixing induced \gls{LFV} rates are predicted far below the sensitivity of current experiments. But if \gls{LFV} would be found on a measurable level, it would imply that another process in a beyond the Standard Model (\gls{BSM}) theory exists, which breaks lepton flavour symmetry. So \gls{LFV} can be used as a probe and a gateway to \gls{BSM} theories. \\

One example of such a theory is the specific super-symmetric extension of the \gls{SM} \cite{SUSYLFV}, which could predict \gls{BSM} \gls{LFV} decays. But a more convenient way to probe \gls{BSM} is to search in a model-independent approach, because a specific \gls{LFV} model is not necessarily realized, but \gls{LFV} could still exist because of another realisation in nature. So to search for example a \gls{LFV} Z decay without any underlying model give rise to the possibility to find \gls{BSM} physics, without constraints due to the specific model dependent phase space. \\

Previous searches for \gls{LFV} in Z bosons were performed and giving first constraints on the branching ratios. In the $e\mu$ final state the best result is measured with the Compact Muon Solenoid (\gls{CMS}) using data from Run-I at $\sqrt{s} = 8$ TeV and set limit on the branching ratios $\text{BR}(Z\to e\mu) < 7.3\cdot 10^{-7}$ \cite{LFVEMU}. The limits in the $e\tau$ and $\mu\tau$ final state are measured at the Large Electron-Positron Collider (\gls{LEP}) \cite{LEP} using data from the run in the years 1991-1993 by the OPAL detector $\text{BR}(Z\to e\tau) < 9.8\cdot 10^{-6}$ \cite{LFVETAU} and from the DELPHI detector $\text{BR}(Z\to \mu\tau) < 1.2\cdot 10^{-5}$ \cite{LFVMUTAU}.
