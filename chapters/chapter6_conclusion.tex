The search for \gls{LFV} in decays of Z bosons is performed for the \gls{LHC} run of 2016 with a center of mass energy of $\sqrt{s} = 13$ TeV and with $35.9$ fb$^{-1}$ data recorded by \gls{CMS}. This analysis evaluated the three possible final states of the direct decay of a Z boson, which violates lepton flavour, namely a pair of electron/muon, electron/$\tau$ lepton and muon/$\tau$ lepton. For the $\tau$ lepton the hadronic decaying $\tau$ have been studied because of the high branching ratio $\text{BR}(\tau \to \nu_{\tau} + \text{hadrons}) = 0.63$. \\

Based on an event selection depending on the final state, well isolated and reconstructed leptons are chosen for the analysis. The kinematic parameters of the leptons are used as an input for a training of a sophisticated \gls{MVA} classifier, in the case of this analysis the \gls{BDT} is the choice. Including the systematic uncertainties, which are adopted from the \gls{SM} $H \to \tau\tau$ analysis, upper expected limits on the branching ratios $\text{BR}(Z \to \text{LFV})$ are set in each final state, using the $\text{CL}_{s}$ method. The expected limit in the $e\mu$ final state showed a clear gain in sensitivity in comparison to previous analysis from \gls{CMS}, in the $e\tau$ and $\mu\tau$ final state the expected sensitivity is competitive with previous analysis from \gls{LEP}. \\

At this moment only the hadronic decay mode of the $\tau$ lepton has been studied in this analysis. The inclusion of the leptonic decay mode of the $\tau$ lepton would result in a gain of sensitivity, although at the moment only different flavour final states of the $\tau$ lepton, namely $e\tau \to e\mu$ and $\mu\tau \to \mu e$ could improve the overall result. Same flavour decay like $e\tau \to ee$ and $\mu\tau \to \mu\mu$ are completely dominated by the \gls{DY} $Z\to ee/\mu\mu$ background. \\

In the $e\tau$ and $\mu\tau$ final state the discrimination power is weaker than in the $e\mu$ final state. A better result of separation performance could be achieved by a deep neural network \cite{DNN}. Since the last 5 years the development of deep neural networks, it evolved into the state-of-the-art method in many disciplines, for example picture recognition \cite{DNN2}. Also in high energy physics deep neural networks show an outstanding performance compared to conventional methods, for example in the analysis of Higgs in association of top quarks \cite{DNN3}. \\

The analysis uses the data set recorded by \gls{CMS} of the year 2016. In fact the \gls{LHC} is running in the Run-II period, which also delivered data in 2017 with a luminosity of 45.9 fb$^{-1}$ and is running in 2018 with a luminosity of 53 fb$^{-1}$ \cite{CMSLUMI1718}. So a total Run-II analysis could use approximately 120 fb$^{-1}$ of recorded events, which would be a increase of statistics by factor of 4 in comparison to this analysis and could improve the overall sensitivity. \\

The final step would be a unblinding of the analysis, to really state, if possible deviation from the \gls{SM} can be seen in data, and to set new constraints on the branching ratio, if no deviations are found. 



 



  




